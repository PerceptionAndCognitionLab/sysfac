\section{Appendix}

Prior specification is needed for $\sigma^2$, the common variance, and for vectors $\bfeta$, $\bfalpha$ and $\bfbeta$.  The prior on $\sigma^2$ is $\sigma^2 \sim \mbox{Inverse Gamma} \left(2,\frac{1}{4}\right)$.   The induced prior on standard deviation has sizable mass above 180 ms, peaks at about 400 ms, and slowly tails off with a fat tail that allows standard deviations as large as 2 seconds.  This prior though informative is sufficiently broad for response times in simple tasks such as those reported here.

The remaining parameters are individual-specific effects, and we chose to model them as  unconstrained random effects with a hierarchical structure:
\begin{eqa*}
\eta_i &\sim& \mbox{Normal}\left(\nu_{1},\delta_{1}\right),\\
\alpha_i &\sim& \mbox{Normal}\left(\nu_{2},\delta_{2}\right),\\
\beta_i &\sim& \mbox{Normal}\left(\nu_{3},\delta_{3}\right),
\end{eqa*}
where $\nu$ and $\delta$ are respective group mean and variances.  Priors on the group mean parameters are
\[
\nu_{1} \sim \mbox{Normal}(2,1),\quad
\nu_{2} \sim \mbox{Normal}\left(0,.032^2\right),\quad 
\nu_{3} \sim \mbox{Normal}\left(0,.032^2\right).
\]

The scale on main effects are tuned, but are reasonable for the size of effects in cognitive psychology.
The priors on the group variance parameters are 
\[
\delta_{k} \sim \mbox{Inverse-Gamma}\left(3,\frac{1}{5}\right), \quad k=1,2,3.
\]

\end{document}
