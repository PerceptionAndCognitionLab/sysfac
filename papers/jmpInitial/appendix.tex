\section{Appendix}

\subsection{Normal Model}

For participant $i = 1,...,I$, level of factor A $j = 1,2$,
level of factor B $k = 1,2$, and replicate $\ell =
1,...,L_{ijk}$, the common framework assumptions are as follows:\\
\begin{eqnarray*}
y_{ijk\ell} &\sim& \mbox{Normal}\left(\mu_{ijk},\sigma^2\right)\\
\mu_{ijk} &=& \eta_i + \alpha_is(j) + \beta_is(k) + \gamma_is(j)s(k)\\
\eta_i &\sim& \mbox{Normal}\left(\nu_{\eta},\delta_{\eta}\right)\\
\alpha_i &\sim& \mbox{Normal}\left(\nu_{\alpha},\delta_{\alpha}\right)\\
\beta_i &\sim& \mbox{Normal}\left(\nu_{\beta},\delta_{\beta}\right)\\
\sigma^2 &\sim& \mbox{Inverse-Gamma}\left(a_{\sigma},b_{\sigma}\right)\\
\nu_{\eta} &\sim& \mbox{Normal}\left(\phi_{\eta},\theta_{\eta}\right)\\
\nu_{\alpha} &\sim& \mbox{Normal}\left(\phi_{\alpha},\theta_{\alpha}\right)\\
\nu_{\beta} &\sim& \mbox{Normal}\left(\phi_{\beta},\theta_{\beta}\right)\\
\delta_{\eta} &\sim& \mbox{Inverse-Gamma}\left(a_{\eta},b_{\eta}\right)\\
\delta_{\alpha} &\sim& \mbox{Inverse-Gamma}\left(a_{\alpha},b_{\alpha}\right)\\
\delta_{\beta} &\sim& \mbox{Inverse-Gamma}\left(a_{\beta},b_{\beta}\right)\\
\end{eqnarray*}

The specific assumptions are as follows:

\subsubsection{Serial Model}
Because the MIC predicted by serial processing is zero, this model
assumes that all interaction parameters, $\gamma_i$ are exactly zero.
Because these parameters are all constant, there is no need for any
further hierarchical specification in this model.

\subsubsection{General Model}
The general model places minimal constraint on the possible range of
the interaction parameters.  For the purposes of this current
research, we will assume that the interaction parameters each follow a
normal distribution with a hierarchical structure described as follows:
\begin{eqnarray*}
\gamma_i &\sim& \mbox{Normal}\left(\nu_{\gamma},\delta_{\gamma}\right)\\
\nu_{\gamma} &\sim& \mbox{Normal}\left(\phi_{\gamma},\theta_{\gamma}\right)\\
\delta_{\gamma} &\sim& \mbox{Inverse-Gamma}\left(a_{\gamma},b_{\gamma}\right)\\
\end{eqnarray*}
where $\phi_{\gamma}$, $\theta_{\gamma}$, $a_{\gamma}$, and
$b_{\gamma}$ are hyperparameters that are specified before analysis.


\subsubsection{Parallel Model}
Because parallel processing predicts negative MICs, this model assumes
that all interaction parameters must be negative.  This is
accomplished by the following restrictions. For Parallel-1:
\begin{eqnarray*}
\gamma_i &\sim& \mbox{Normal}^-(\nu_{\gamma},\delta_{\gamma})
\end{eqnarray*}
with the above distribution being the half-normal distribution "centered" at $\nu_{\gamma}$ and restricted to negative values. 
For Parallel-2, the interaction effect is treated as being identical across all participants:
\begin{eqnarray*}
\gamma_i &=& \gamma_0 \forall i\\
\gamma_0 &\sim& \mbox{Normal}^-(\nu_{\gamma},\delta_{\gamma})
\end{eqnarray*}
This limitation to the support set constitutes the sign constraint needed for this model and only allows interaction parameters to equal zero in order to allow for Savage-Dickey estimation.  The priors placed on $\nu_{\gamma}$ and $\delta_{\gamma}$ for the parallel model are identical to those used in the General model.


\subsubsection{Coactive Model}
Because coactive processing predicts positive MICs, this model assumes
that all interaction parameters must be positive.  This is
accomplished by the following restrictions. For Coactive-1:
\begin{eqnarray*}
\gamma_i &\sim& \mbox{Normal}^+(\nu_{\gamma},\delta_{\gamma})
\end{eqnarray*}
with the above distribution being the half-normal distribution "centered" at $\nu_{\gamma}$ and restricted to positive values. 
For Coactive-2, the interaction effect is treated as being identical across all participants:
\begin{eqnarray*}
\gamma_i &=& \gamma_0 \forall i \\
\gamma_0 &\sim& \mbox{Normal}^+(\nu_{\gamma},\delta_{\gamma})
\end{eqnarray*}
This limitation to the support set constitutes the sign constraint needed for this model and only allows interaction parameters to equal zero in order to allow for Savage-Dickey estimation.  The priors placed on $\nu_{\gamma}$ and $\delta_{\gamma}$ for the parallel model are identical to those used in the General model.


\subsubsection{Joint Distribution and Posterior Derivation of Common Parameters}
Because the common parameters are independent of the specific
parameters and only depend on them when they are assumed to be known,
the priors for the model-specific parameters are effectively
normalizing constants that can be dropped during derivation up to
proportionality.  As such, those distributions are of no interest to
the current joint analysis and will be omitted.  Consequently, the
joint distribution of all common parameters is as follows:\\
\begin{eqnarray*}
&f\left(Y,\vec{\eta},\vec{\alpha},\vec{\beta},\vec{\gamma},\sigma^2,
\nu_{\eta},\nu_{\alpha},\nu_{\beta},\nu_{\gamma},\delta_{\eta},\delta_{\alpha},
\delta_{\beta},\delta_{\gamma}\right)&\\
&\displaystyle{\propto\prod_{i}\prod_{j}\prod_{k}\prod_{l}\left(2\pi\sigma^2\right)^{-\frac{1}{2}}\exp\left(-\frac{\left(y_{ijk\ell}-\mu_{ijk}\right)^2}{2\sigma^2}\right)
\times\prod_i\left(2\pi\delta_{\eta}\right)^{-\frac{1}{2}}\exp\left(-\frac{\left(\eta_i-\nu_{\eta}\right)^2}{2\delta_{\eta}}\right)}&\\
&\displaystyle{\times\prod_i\left(2\pi\delta_{\alpha}\right)^{-\frac{1}{2}}\exp\left(-\frac{\left(\alpha_i-\nu_{\alpha}\right)^2}{2\delta_{\alpha}}\right)
\times\prod_i\left(2\pi\delta_{\beta}\right)^{-\frac{1}{2}}\exp\left(-\frac{\left(\beta_i-\nu_{\beta}\right)^2}{2\delta_{\beta}}\right)}&\\
&\displaystyle{\times\frac{b_{\sigma}^{a_{\sigma}}}{\Gamma\left(a_{\sigma}\right)}\left(\sigma^2\right)^{-a_{\sigma}-1}\exp\left(-\frac{b_{\sigma}}{\sigma^2}\right)
\times\left(2\pi\theta_{\eta}\right)^{-\frac{1}{2}}\exp\left(-\frac{\left(\nu_{\eta}-\phi_{\eta}\right)^2}{2\theta_{\eta}}\right)}&\\
&\displaystyle{\times\left(2\pi\theta_{\alpha}\right)^{-\frac{1}{2}}\exp\left(-\frac{\left(\nu_{\alpha}-\phi_{\alpha}\right)^2}{2\theta_{\alpha}}\right)
\times\left(2\pi\theta_{\beta}\right)^{-\frac{1}{2}}\exp\left(-\frac{\left(\nu_{\beta}-\phi_{\beta}\right)^2}{2\theta_{\beta}}\right)}&\\
&\displaystyle{
\times\frac{b_{\eta}^{a_{\eta}}}{\Gamma\left(a_{\eta}\right)}\delta_{\eta}^{-a_{\eta}-1}\exp\left(-\frac{b_{\eta}}{\delta_{\eta}}\right)
%}&\\
%&\displaystyle{
\times\frac{b_{\alpha}^{a_{\alpha}}}{\Gamma\left(a_{\alpha}\right)}\delta_{\alpha}^{-a_{\alpha}-1}\exp\left(-\frac{b_{\alpha}}{\delta_{\alpha}}\right)
\times\frac{b_{\beta}^{a_{\beta}}}{\Gamma\left(a_{\beta}\right)}\delta_{\beta}^{-a_{\beta}-1}\exp\left(-\frac{b_{\beta}}{\delta_{\beta}}\right)}&\\
%%%%%%%%%%%%%%%%%%%%%%%%%%%%%%%%%%%%%%%%%%%%%%%
%%%%%%%%%%%%%%%%%%%%%%%%%%%%%%%%%%%%%%%%%%%%%%%
&\displaystyle{\propto\left(2\pi\sigma^2\right)^{-\frac{\sum_i\sum_j\sum_kL_{ijk}}{2}}\exp\left(-\frac{\sum_i\sum_j\sum_k\sum_l\left(y_{ijk\ell}-\eta_i-\alpha_i\left(-1\right)^{j}-\beta_i\left(-1\right)^{k}-\gamma_i\left(-1\right)^{j+k}\right)^2}{2\sigma^2}\right)}&\\
&\displaystyle{\times\left(2\pi\delta_{\eta}\right)^{-\frac{I}{2}}\exp\left(-\frac{\sum_i\left(\eta_i-\nu_{\eta}\right)^2}{2\delta_{\eta}}\right)
\times\left(2\pi\delta_{\alpha}\right)^{-\frac{I}{2}}\exp\left(-\frac{\sum_i\left(\alpha_i-\nu_{\alpha}\right)^2}{2\delta_{\alpha}}\right)}&\\
&\displaystyle{\times\left(2\pi\delta_{\beta}\right)^{-\frac{I}{2}}\exp\left(-\frac{\sum_i\left(\beta_i-\nu_{\beta}\right)^2}{2\delta_{\beta}}\right)
\times\frac{b_{\sigma}^{a_{\sigma}}}{\Gamma(a_{\sigma})}\left(\sigma^2\right)^{-a_{\sigma}-1}\exp\left(-\frac{b_{\sigma}}{\sigma^2}\right)}&\\
&\displaystyle{\times\left(2\pi\theta_{\eta}\right)^{-\frac{1}{2}}\exp\left(-\frac{\left(\nu_{\eta}-\phi_{\eta}\right)^2}{2\theta_{\eta}}\right)
\times\left(2\pi\theta_{\alpha}\right)^{-\frac{1}{2}}\exp\left(-\frac{\left(\nu_{\alpha}-\phi_{\alpha}\right)^2}{2\theta_{\alpha}}\right)}&\\
&\displaystyle{\times\left(2\pi\theta_{\beta}\right)^{-\frac{1}{2}}\exp\left(-\frac{\left(\nu_{\beta}-\phi_{\beta}\right)^2}{2\theta_{\beta}}\right)
\times\frac{b_{\eta}^{a_{\eta}}}{\Gamma(a_{\eta})}\delta_{\eta}^{-a_{\eta}-1}\exp\left(-\frac{b_{\eta}}{\delta_{\eta}}\right)}&\\
&\displaystyle{\times\frac{b_{\alpha}^{a_{\alpha}}}{\Gamma(a_{\alpha})}\delta_{\alpha}^{-a_{\alpha}-1}\exp\left(-\frac{b_{\alpha}}{\delta_{\alpha}}\right)
\times\frac{b_{\beta}^{a_{\beta}}}{\Gamma(a_{\beta})}\delta_{\beta}^{-a_{\beta}-1}\exp\left(-\frac{b_{\beta}}{\delta_{\beta}}\right)}&\\
\end{eqnarray*}

This result is proportional because the distributions for $\gamma_i$,
$\nu_{\gamma}$, and $\delta_{\gamma}$ are not inlcuded in the above
formulation.  This is because those distributions do not contain any
factors of the above common parameters and, consequently, are of
constant value with respect to them.  Thus, using the above joint
distribution to solve the posterior distribution of any common
parameter is equivalent to using the full joint for any given model.
With that in mind, the posterior distributions for the common
parameters are given as follows:

$\eta_i$:
\begin{eqnarray*}
f(\eta_i|\cdot)&\propto&\exp\left(-\frac{\sum_j\sum_k\sum_l\left(y_{ijk\ell}-\eta_i-\alpha_i\left(-1\right)^{j}-\beta_i\left(-1\right)^{k}-\gamma_i\left(-1\right)^{j+k}\right)^2}{2\sigma^2}\right)\\
&&\times\exp\left(-\frac{\left(\eta_i-\nu_{\eta}\right)^2}{2\delta_{\eta}}\right)\\
&*&\left(y_{ijk\ell}-\eta_i-\alpha_i\left(-1\right)^{j}-\beta_i\left(-1\right)^{k}-\gamma_i\left(-1\right)^{j+k}\right)^2\\
&&=y_{ijk\ell}^2+\eta_i^2+\alpha_i^2(-1)^{2j}+\beta_i^2(-1)^{2k}+\gamma_i^2(-1)^{2j+2k}-2y_{ijk\ell}\eta_i\\
&&-2y_{ijk\ell}\alpha_i(-1)^j-2y_{ijk\ell}\beta_i(-1)^k-2y_{ijk\ell}\gamma_i(-1)^{j+k}+2\eta_i\alpha_i(-1)^j\\
&&+2\eta_i\beta_i(-1)^k+2\eta_i\gamma_i(-1)^{j+k}+2\alpha_i\beta_i(-1)^{j+k}+2\alpha_i\gamma_i(-1)^{2j+k}\\
&&+2\beta_i\gamma_i(-1)^{j+2k}\\
&&=\eta_i^2-2y_{ijk\ell}\eta_i+2\eta_i\alpha_i(-1)^j+2\eta_i\beta_i(-1)^k+2\eta_i\gamma_i(-1)^{j+k}+y_{ijk\ell}^2\\
&&+\alpha_i^2(-1)^{2j}+\beta_i^2(-1)^{2k}+\gamma_i^2(-1)^{2j+2k}-2y_{ijk\ell}\alpha_i(-1)^j-2y_{ijk\ell}\beta_i(-1)^k\\
&&-2y_{ijk\ell}\gamma_i(-1)^{j+k}+2\alpha_i\beta_i(-1)^{j+k}+2\alpha_i\gamma_i(-1)^{2j+k}+2\beta_i\gamma_i(-1)^{j+2k}\\
\Rightarrow f(\eta_i|\cdot)&\propto&\exp\left(-\frac{\sum_j\sum_k\sum_l\eta_i^2-2y_{ijk\ell}\eta_i+2\eta_i\alpha_i(-1)^j+2\eta_i\beta_i(-1)^k+2\eta_i\gamma_i(-1)^{j+k}}{2\sigma^2}\right)\\
&&\times\exp\left(-\frac{\eta_i^2-2\eta_i\nu_{\eta}}{2\delta_{\eta}}\right)\\
&=&\exp\left(-\frac{L_{i\cdot\cdot}\eta_i^2-2\eta_i\left(\sum_j\sum_k\sum_ly_{ijk\ell}-\sum_j\alpha_i(-1)^jL_{ij\cdot}-\sum_k\beta_i(-1)^kL_{i\cdot k}\right.}{2\sigma^2}\right.\\
&&\left.\frac{\left.-\sum_j\sum_k\gamma_i(-1)^{j+k}L_{ijk}\right)}{}\right)\times\exp\left(-\frac{\eta_i^2-2\eta_i\nu_{\eta}}{2\delta_{\eta}}\right)\\
\end{eqnarray*}
for $L_{i\cdot k} = \sum_jL_{ijk}$, $L_{ij\cdot} = \sum_kL_{ijk}$, and $L_{i\cdot\cdot} = \sum_j\sum_kL_{ijk}$\\
\begin{eqnarray*}
&=&\exp\left(-\frac{1}{2}\left[\eta_i^2\left(\frac{L_{i\cdot\cdot}}{\sigma^2}+\frac{1}{\delta_{\eta}}\right)-2\eta_i\left(\frac{L_{i\cdot\cdot}\bar{y}_{ijk\ell}-L_{i\alpha}\alpha_i-L_{i\beta}\beta_i-L_{i\gamma}\gamma_i}{\sigma^2}+\frac{\nu_{\eta}}{\delta_{\eta}}\right)\right]\right)\\
\end{eqnarray*}
for $L_{i\alpha}=L_{i2\cdot}-L_{i1\cdot}$, $L_{i\beta}=L_{i\cdot 2}-L_{i\cdot 1}$, and $L_{i\gamma}=L_{i22}+L_{i11}-L_{i12}-L_{i21}$\\
\begin{eqnarray*}
&=&\exp\left(-\frac{1}{2}\left[\frac{L_{i\cdot\cdot}}{\sigma^2}+\frac{1}{\delta_{\eta}}\right]\left[\eta_i^2-2\eta_i\left(\frac{L_{i\cdot\cdot}\bar{y}_{ijk\ell}-L_{i\alpha}\alpha_i-L_{i\beta}\beta_i-L_{i\gamma}\gamma_i}{\sigma^2}+\frac{\nu_{\eta}}{\delta_{\eta}}\right)\right.\right.\\
&&\left.\left.\left(\frac{L_{i\cdot\cdot}}{\sigma^2}+\frac{1}{\delta_{\eta}}\right)^{-1}\right]\right)\\
\Rightarrow \eta_i|\cdot &\sim& \mbox{Normal}\left(\left(\frac{L_{i\cdot\cdot}\bar{y}_{ijk\ell}-L_{i\alpha}\alpha_i-L_{i\beta}\beta_i-L_{i\gamma}\gamma_i}{\sigma^2}+\frac{\nu_{\eta}}{\delta_{\eta}}\right)\left(\frac{L_{i\cdot\cdot}}{\sigma^2}+\frac{1}{\delta_{\eta}}\right)^{-1},\left(\frac{L_{i\cdot\cdot}}{\sigma^2}+\frac{1}{\delta_{\eta}}\right)^{-1}\right)\\
\end{eqnarray*}
Ideally, $L_{i\alpha}$, $L_{i\beta}$, and $L_{i\gamma}$ would all
equal 0 and all non-data-related terms would fall out of the
posterior, but these factors are being included in the event that one
or more of these factorial manipulations isn't equally represented,
whether this failure to include is a result of data cleaning, poor
experiment coding, or some unforseen accident.\\

$\alpha_{i}$:
Although the terms kept in the exp function are different, specifically
\begin{eqnarray*}
&&\alpha_i^2(-1)^{2j}-2y_{ijk\ell}\alpha_i(-1)^j+2\eta_i\alpha_i(-1)^j+2\alpha_i\beta_i(-1)^{j+k}+2\alpha_i\gamma_i(-1)^{2j+k}\\ 
&=& \alpha_i^2-2y_{ijk\ell}\alpha_i(-1)^j+2\eta_i\alpha_i(-1)^j+2\alpha_i\beta_i(-1)^{j+k}+2\alpha_i\gamma_i(-1)^k)\\
&\Rightarrow&\sum_j\sum_k\sum_l\alpha_i^2-2y_{ijk\ell}\alpha_i(-1)^j+2\eta_i\alpha_i(-1)^j+2\alpha_i\beta_i(-1)^{j+k}+2\alpha_i\gamma_i(-1)^{k}\\
&=&L_{i\cdot\cdot}\alpha_i^2-2\alpha_i\left(\sum_j\sum_k\sum_ly_{ijk\ell}(-1)^j-L_{i\alpha}\eta_i-L_{i\gamma}\beta_i-L_{i\beta}\gamma_i\right)\\
\end{eqnarray*}
the derivation for $f(\alpha_i|\cdot)$ is similar to that for
$f(\eta_i|\cdot)$.  As such, only the posterior will be listed here as
a space-saving measure.
\begin{eqnarray*}
\alpha_i|\cdot&\sim&\mbox{Normal}\left(\left(\frac{\sum_j\sum_k\sum_l\bar{y}_{ijk\ell}(-1)^j-L_{i\alpha}\eta_i-L_{i\gamma}\beta_i-L_{i\beta}\gamma_i}{\sigma^2}+\frac{\nu_{\eta}}{\delta_{\eta}}\right)\left(\frac{L_{i\cdot\cdot}}{\sigma^2}+\frac{1}{\delta_{\eta}}\right)^{-1}\right.\\
&&\left.,\left(\frac{L_{i\cdot\cdot}}{\sigma^2}+\frac{1}{\delta_{\eta}}\right)^{-1}\right)\\
\end{eqnarray*}

$\beta_{i}$:\\
Similarly to $\alpha_i$,
\begin{eqnarray*}
&&\beta_i^2(-1)^{2k}-2y_{ijk\ell}\beta_i(-1)^k+2\eta_i\beta_i(-1)^k+2\alpha_i\beta_i(-1)^{j+k}+2\beta_i\gamma_i(-1)^{j+2k}\\
&=&\beta_i^2-2y_{ijk\ell}\beta_i(-1)^k+2\eta_i\beta_i(-1)^k+2\alpha_i\beta_i(-1)^{j+k}+2\beta_i\gamma_i(-1)^j\\
&\Rightarrow&\sum_j\sum_k\sum_l\beta_i^2-2y_{ijk\ell}\beta_i(-1)^k+2\eta_i\beta_i(-1)^k+2\alpha_i\beta_i(-1)^{j+k}+2\beta_i\gamma_i(-1)^j\\
&=&L_{i\cdot\cdot}\beta_i^2-2\beta\left(\sum_j\sum_k\sum_ly_{ijk\ell}(-1)^k-L_{i\beta}\eta_i-L_{i\gamma}\alpha_i-L_{i\alpha}\gamma_i\right)\\
\end{eqnarray*}
comprises the terms kept inside the exp function, with the remaining
derivation of $f(\beta_i|\cdot)$ being similar to the derivations of
$f(\eta_i|\cdot)$ and $f(\alpha_i|\cdot)$.  Thus, the posterior for
$\beta_i|\cdot$ is,
\begin{eqnarray*}
\beta_i|\cdot&\sim&\mbox{Normal}\left(\left(\frac{\sum_j\sum_k\sum_l\bar{y}_{ijk\ell}(-1)^k-L_{i\beta}\eta_i-L_{i\gamma}\alpha_i-L_{i\alpha}\gamma_i}{\sigma^2}+\frac{\nu_{\eta}}{\delta_{\eta}}\right)\left(\frac{L_{i\cdot\cdot}}{\sigma^2}+\frac{1}{\delta_{\eta}}\right)^{-1}\right.\\
&&\left.,\left(\frac{L_{i\cdot\cdot}}{\sigma^2}+\frac{1}{\delta_{\eta}}\right)^{-1}\right)\\
\end{eqnarray*}

$\sigma^2$:\\
\begin{eqnarray*}
f(\sigma^2|\cdot)&\propto&\left(2\pi\sigma^2\right)^{-\frac{\sum_i\sum_j\sum_kL_{ijk}}{2}}\exp\left(-\frac{\sum_i\sum_j\sum_k\sum_l\left(y_{ijk\ell}-\mu_{ijk}\right)^2}{2\sigma^2}\right)\\
&&\times\frac{b_{\sigma}^{a_{\sigma}}}{\Gamma\left(a_{\sigma}\right)}\left(\sigma^2\right)^{-a_{\sigma}-1}\exp\left(-\frac{b_{\sigma}}{\sigma^2}\right)\\
&\propto&\left(\sigma^2\right)^{-\frac{L_{\cdots}}{2}-a_{\sigma}-1}\exp\left(-\frac{1}{\sigma^2}\left[\frac{\sum_i\sum_j\sum_k\sum_l\left(y_{ijk\ell}-\mu_{ijk}\right)^2}{2}+b_{\sigma}\right]\right)\\
\end{eqnarray*}
for $L_{\cdots} = \sum_i\sum_j\sum_kL_{ijk}$.\\
\begin{eqnarray*}
\Rightarrow \sigma^2|\cdot &\sim& \mbox{Inverse-Gamma}\left(\frac{L_{\cdots}}{2}+a_{\sigma},\frac{\sum_i\sum_j\sum_k\sum_l\left(y_{ijk\ell}-\mu_{ijk}\right)^2}{2}+b_{\sigma}\right)\\
\end{eqnarray*}
For the given priors,
\begin{eqnarray*}
\sigma^2|\cdot &\sim& \mbox{Inverse-Gamma}\left(\frac{L_{\cdots}}{2}+2,\frac{\sum_i\sum_j\sum_k\sum_l\left(y_{ijk\ell}-\mu_{ijk}\right)^2}{2}+\frac{1}{4}\right)\\
\end{eqnarray*}

$\nu_{\eta}$:\\
\begin{eqnarray*}
f(\nu_{\eta}|\cdot)&\propto&\exp\left(-\frac{\sum_i(\eta_i-\nu_{\eta})^2}{2\delta_{\eta}}\right)\times\exp\left(-\frac{(\nu_{\eta}-\phi_{\eta})^2}{2\theta_{\eta}}\right)\\
&=&\exp\left(-\frac{\sum_i(\eta_i^2-2\eta_i\nu_{\eta}+\nu_{\eta}^2)}{2\delta_{\eta}}\right)\times\exp\left(-\frac{\nu_{\eta}^2-2\nu_{\eta}\phi_{\eta}+\phi_{\eta}^2}{2\theta_{\eta}}\right)\\
&\propto&\exp\left(-\frac{1}{2}\left[\frac{-2\nu_{\eta}\sum_i\eta_i+I\nu_{\eta}^2}{\delta_{\eta}}+\frac{\nu_{\eta}^2-2\nu_{\eta}\phi_{\eta}}{\theta_{\eta}}\right]\right)\\
&=&\exp\left(-\frac{1}{2}\left[\nu_{\eta}^2\left(\frac{I}{\delta_{\eta}}+\frac{1}{\theta_{\eta}}\right)-2\nu_{\eta}\left(\frac{I\bar{\eta}}{\delta_{\eta}}+\frac{\phi_{\eta}}{\theta_{\eta}}\right)\right]\right)\\
&=&\exp\left(-\frac{1}{2}\left[\frac{I}{\delta_{\eta}}+\frac{1}{\theta_{\eta}}\right]\left[\nu_{\eta}^2-2\nu_{\eta}\left(\frac{I\bar{\eta}}{\delta_{\eta}}+\frac{\phi_{\eta}}{\theta_{\eta}}\right)\left(\frac{I}{\delta_{\eta}}+\frac{1}{\theta_{\eta}}\right)^{-1}\right]\right)\\
\Rightarrow \nu_{\eta}|\cdot&\sim&\mbox{Normal}\left(\left(\frac{I\bar{\eta}}{\delta_{\eta}}+\frac{\phi_{\eta}}{\theta_{\eta}}\right)\left(\frac{I}{\delta_{\eta}}+\frac{1}{\theta_{\eta}}\right)^{-1},\left(\frac{I}{\delta_{\eta}}+\frac{1}{\theta_{\eta}}\right)^{-1}\right)\\
\end{eqnarray*}
For the given priors,
\begin{eqnarray*}
\nu_{\eta}|\cdot &\sim& \mbox{Normal}\left(\left(\frac{I\bar{\eta}}{\delta_{\eta}}+2\right)\left(\frac{I}{\delta_{\eta}}+1\right)^{-1},\left(\frac{I}{\delta_{\eta}}+1\right)^{-1}\right)\\
\end{eqnarray*}

$\nu_{\alpha}$:\\
The posterior for $\nu_{\alpha}$ is derived similarly to that for
$\nu_{\eta}$.  As such, the derivation will not be included in the
interest of saving space.
\begin{eqnarray*}
\nu_{\alpha}|\cdot&\sim&\mbox{Normal}\left(\left(\frac{I\bar{\alpha}}{\delta_{\alpha}}+\frac{\phi_{\alpha}}{\theta_{\alpha}}\right)\left(\frac{I}{\delta_{\alpha}}+\frac{1}{\theta_{\alpha}}\right)^{-1},\left(\frac{I}{\delta_{\alpha}}+\frac{1}{\theta_{\alpha}}\right)^{-1}\right)\\
\end{eqnarray*}
For the given priors,
\begin{eqnarray*}
\nu_{\alpha}|\cdot&\sim&\mbox{Normal}\left(\left(\frac{I\bar{\alpha}}{\delta_{\alpha}}\right)\left(\frac{I}{\delta_{\alpha}}+4\right)^{-1},\left(\frac{I}{\delta_{\alpha}}+4\right)^{-1}\right)\\
\end{eqnarray*}

$\nu_{\beta}$:\\
The posterior for $\nu_{\beta}$ is derived similarly to that for
$\nu_{\eta}$.  As such, the derivation will not be included in the
interest of saving space.
\begin{eqnarray*}
\nu_{\beta}|\cdot&\sim&\mbox{Normal}\left(\left(\frac{I\bar{\beta}}{\delta_{\beta}}+\frac{\phi_{\beta}}{\theta_{\beta}}\right)\left(\frac{I}{\delta_{\beta}}+\frac{1}{\theta_{\beta}}\right)^{-1},\left(\frac{I}{\delta_{\beta}}+\frac{1}{\theta_{\beta}}\right)^{-1}\right)\\
\end{eqnarray*}
For the given priors,
\begin{eqnarray*}
\nu_{\beta}|\cdot&\sim&\mbox{Normal}\left(\left(\frac{I\bar{\beta}}{\delta_{\beta}}\right)\left(\frac{I}{\delta_{\beta}}+4\right)^{-1},\left(\frac{I}{\delta_{\beta}}+4\right)^{-1}\right)\\
\end{eqnarray*}

$\delta_{\eta}$:\\
\begin{eqnarray*}
f(\delta_{\eta}|\cdot)&\propto&\delta_{\eta}^{-\frac{I}{2}}\exp\left(-\frac{\sum_i(\eta_i-\nu_{\eta})^2}{2\delta_{\eta}}\right)\delta_{\eta}^{-a_{\eta}-1}\exp\left(-\frac{b_{\eta}}{\delta_{\eta}}\right)\\
&=&\delta_{\eta}^{-\frac{I}{2}-a_{\eta}-1}\exp\left(-\frac{1}{\delta_{\eta}}\left[\frac{\sum_i(\eta_i-\nu_{\eta})^2}{2}+b_{\eta}\right]\right)\\
\Rightarrow \delta_{\eta}|\cdot&\sim&\mbox{Inverse-Gamma}\left(\frac{I}{2}+a_{\eta},\frac{\sum_i(\eta_i-\nu_{\eta})^2}{2}+b_{\eta}\right)\\
\end{eqnarray*}
For the given priors,
\begin{eqnarray*}
\delta_{\eta}|\cdot&\sim&\mbox{Inverse-Gamma}\left(\frac{I}{2}+2,\frac{\sum_i(\eta_i-\nu_{\eta})^2}{2}+\frac{1}{4}\right)\\
\end{eqnarray*}

$\delta_{\alpha}$:\\
The posterior for $\delta_{\alpha}$ is derived similarly to that for
$\delta_{\eta}$.  As such, the derivation will not be included in the
interest of saving space.
\begin{eqnarray*}
\delta_{\alpha}|\cdot&\sim&\mbox{Inverse-Gamma}\left(\frac{I}{2}+a_{\alpha},\frac{\sum_i(\alpha_i-\nu_{\alpha})^2}{2}+b_{\alpha}\right)\\
\end{eqnarray*}
For the given priors,
\begin{eqnarray*}
\delta_{\alpha}|\cdot&\sim&\mbox{Inverse-Gamma}\left(\frac{I}{2}+2,\frac{\sum_i(\alpha_i-\nu_{\alpha})^2}{2}+\frac{1}{4}\right)\\
\end{eqnarray*}

$\delta_{\beta}$:\\
The posterior for $\delta_{\beta}$ is derived similarly to that for
$\delta_{\eta}$.  As such, the derivation will not be included in the
interest of saving space.
\begin{eqnarray*}
\delta_{\beta}|\cdot&\sim&\mbox{Inverse-Gamma}\left(\frac{I}{2}+a_{\beta},\frac{\sum_i(\beta_i-\nu_{\beta})^2}{2}+b_{\beta}\right)\\
\end{eqnarray*}
For the given priors,
\begin{eqnarray*}
\delta_{\beta}|\cdot&\sim&\mbox{Inverse-Gamma}\left(\frac{I}{2}+2,\frac{\sum_i(\beta_i-\nu_{\beta})^2}{2}+\frac{1}{4}\right)\\
\end{eqnarray*}


\subsubsection{Specific Posterior Derivations}

The model-specific posteriors are as follows.\\

\subsubsection{Serial Model} 

In this model, all $\gamma_i$ are equal to 0. Consequently, there is
no need for estimation of either $\nu_{\gamma}$ or $\delta_{\gamma}$
parameters, as they effectively do not exist for this model.\\


\subsubsection{General Model}

In this model, the prior distribution over each $\gamma_i$ parameter
is similar to the priors for the $\eta_i$, $\alpha_i$, and $\beta_i$
parameters.  As such, the posterior derivations for $\gamma_i$,
$\nu_{\gamma}$, and $\delta_{\gamma}$ are similar to those for
$\eta_i$, $\nu_{\eta}$, $\delta_{\eta}$, etc. and, consequently, are
withheld in this section for space, with only the posteriors defined
as follows.\\

$\gamma_i$:\\
\begin{eqnarray*}
\gamma_i|\dots&\sim&\mbox{Normal}\left(\left(\frac{\sum_j\sum_k\sum_l\bar{y}_{ijk\ell}(-1)^{j+k}-L_{i\gamma}\eta_i-L_{i\beta}\alpha_i-L_{i\alpha}\beta_i}{\sigma^2}+\frac{\nu_{\gamma}}{\delta_{\gamma}}\right)\right.\\
&&\left.\times\left(\frac{L_{i\cdot\cdot}}{\sigma^2}+\frac{1}{\delta_{\gamma}}\right)^{-1},\left(\frac{L_{i\cdot\cdot}}{\sigma^2}+\frac{1}{\delta_{\gamma}}\right)^{-1}\right)\\
\end{eqnarray*}

$\nu_{\gamma}$:\\
\begin{eqnarray*}
\nu_{\gamma}|\dots&\sim&\mbox{Normal}\left(\left(\frac{I\bar{\gamma}}{\delta_{\gamma}}+\frac{\phi_{\gamma}}{\theta_{\gamma}}\right)\left(\frac{I}{\delta_{\gamma}}+\frac{1}{\theta_{\gamma}}\right)^{-1},\left(\frac{I}{\delta_{\gamma}}+\frac{1}{\theta_{\gamma}}\right)^{-1}\right)\\
\end{eqnarray*}
For the given priors,
\begin{eqnarray*}
\nu_{\gamma}|\dots&\sim&\mbox{Normal}\left(\left(\frac{I\bar{\gamma}}{\delta_{\gamma}}\right)\left(\frac{I}{\delta_{\gamma}}+4\right)^{-1},\left(\frac{I}{\delta_{\gamma}}+4\right)^{-1}\right)\\
\end{eqnarray*}

$\delta_{\gamma}$:\\
\begin{eqnarray*}
\delta_{\gamma}|\dots&\sim&\mbox{Inverse-Gamma}\left(\frac{I}{2}+a_{\gamma},\frac{\sum_i(\gamma_i-\nu_{\gamma})^2}{2}+b_{\gamma}\right)\\
\end{eqnarray*}
For the given priors,
\begin{eqnarray*}
\delta_{\gamma}|\dots&\sim&\mbox{Inverse-Gamma}\left(\frac{I}{2}+2,\frac{\sum_i(\gamma_i-\nu_{\gamma})^2}{2}+\frac{1}{4}\right)\\
\end{eqnarray*}


\subsubsection{Parallel Model}

In this model, all $\gamma_i$ parameters follow a Truncated-Normal
distribution that is limited between $-\infty$ and 0.  The density for
$\gamma_i$ in this case is:
\begin{eqnarray*}
f(\gamma_i)&=&(2\pi\delta_{\gamma})^{-\frac{1}{2}}\exp\left(-\frac{(\gamma_i-\nu_{\gamma})^2}{2\delta_{\gamma}}\right)\mathcal{I}\left(\gamma_i<0\right)\Phi\left(\frac{0-\nu_{\gamma}}{\sqrt{\delta_{\gamma}}}\right)^{-1}\\
&=&(2\pi\delta_{\gamma})^{-\frac{1}{2}}\exp\left(-\frac{(\gamma_i-\nu_{\gamma})^2}{2\delta_{\gamma}}\right)\mathcal{I}\left(\gamma_i<0\right)\Phi\left(\frac{-\nu_{\gamma}}{\sqrt{\delta_{\gamma}}}\right)^{-1}\\
\end{eqnarray*}
where $\mathcal{I}\left(\gamma_i<0\right)$ is an indicator function
that is equal to 1 when $\gamma_i$ is less than 0 and is equal to 0
otherwise and $\Phi$ is the cumulative density function of a standard
normal distribution.\\

$\gamma_i$:\\
\begin{eqnarray*}
f(\gamma_i|\dots)&\propto&\exp\left(-\frac{\sum_j\sum_k\sum_l\gamma_i^2-2y_{ijk\ell}\gamma_i(-1)^{j+k}+2\eta_i\gamma_i(-1)^{j+k}+2\alpha_i\gamma_i(-1)^k}{2\sigma^2}\right.\\
&&\left.\frac{+2\beta_i\gamma_i(-1)^j}{}\right)\times\exp\left(-\frac{\gamma_i^2-2\gamma_i\nu_{\gamma}+\nu_{\gamma}^2}{2\delta_{\gamma}}\right)\mathcal{I}\left(\gamma_i<0\right)\\
&\propto&\exp\left(-\frac{1}{2}\left[\gamma_i^{2}\left(\frac{L_{i\cdot\cdot}}{\sigma^2}+\frac{1}{\delta_{\gamma}}\right)-2\gamma_i\left(\frac{\sum_j\sum_k\sum_ly_{ijk\ell}-L_{i\gamma}\eta_i-L_{i\beta}\alpha_i-L_{i\alpha}\beta_i}{\sigma^2}\right.\right.\right.\\
&&\left.\left.\left.+\frac{\nu_{\gamma}}{\delta_{\gamma}}\right)\right]\right)\mathcal{I}\left(\gamma_i<0\right)\\
\Rightarrow\gamma_i|\dots3&\sim&\mbox{Truncated-Normal}\left(\nu_{\gamma}',\delta_{\gamma}',\left\{-\infty,\cdots,0\right\}\right)
\end{eqnarray*}
for $\displaystyle{\delta_{\gamma}' = \left(\frac{L_{i\cdot\cdot}}{\sigma^2}+\frac{1}{\delta_{\gamma}}\right)^{-1}}$ and $\displaystyle{\nu_{\gamma}' = \left(\frac{\sum_j\sum_k\sum_ly_{ijk\ell}-L_{i\gamma}\eta_i-L_{i\beta}\alpha_i-L_{i\alpha}\beta_i}{\sigma^2}+\frac{\nu_{\gamma}}{\delta_{\gamma}}\right)\delta_{\gamma}'}$\\

$\nu_{\gamma}$:\\
\begin{eqnarray*}
f(\nu_{\gamma}|\dots)&\propto&\exp\left(-\frac{\sum_i\left(\gamma_i-\nu_{\gamma}\right)^2}{2\delta_{\gamma}}\right)\Phi\left(\frac{-\nu_{\gamma}}{\sqrt{\delta_{\gamma}}}\right)^{-I}\exp\left(-\frac{(\nu_{\gamma}-\phi_{\gamma})^2}{2\theta_{\gamma}}\right)\\
&=&\exp\left(-\frac{1}{2}\left[\frac{\sum_i\gamma_i^2-2\gamma_i\nu_{\gamma}+\nu_{\gamma}^2}{\delta_{\gamma}}+\frac{\nu_{\gamma}^2-2\nu_{\gamma}\phi_{\gamma}+\phi_{\gamma}^2}{\theta_{\gamma}}\right]\right)\Phi\left(\frac{-\nu_{\gamma}}{\sqrt{\delta_{\gamma}}}\right)^{-I}\\
&\propto&\exp\left(-\frac{1}{2}\left[\frac{-2\nu_{\gamma}I\bar{\gamma}+I\nu_{\gamma}^2}{\delta_{\gamma}}+\frac{\nu_{\gamma}^2-2\nu_{\gamma}\phi_{\gamma}}{\theta_{\gamma}}\right]\right)\Phi\left(\frac{-\nu_{\gamma}}{\sqrt{\delta_{\gamma}}}\right)^{-I}\\
&=&\exp\left(-\frac{1}{2}\left[\nu_{\gamma}^2\left(\frac{I}{\delta_{\gamma}}+\frac{1}{\theta_{\gamma}}\right)-2\nu_{\gamma}\left(\frac{I\bar{\gamma}}{\delta_{\gamma}}+\frac{\phi_{\gamma}}{\theta_{\gamma}}\right)\right]\right)\Phi\left(\frac{-\nu_{\gamma}}{\sqrt{\delta_{\gamma}}}\right)^{-I}\\
\end{eqnarray*}
This distribution is not a Normal distribution.  Thus, the Normal
distribution is not a conjugate prior for this parameter.  This
distribution is also not of a commonly known form, as no common
distribution has a density composed of a product of a Standard Normal
density function scales by a Standard Normal CDF.  Consequently,
Metropolis-Hastings sampling is needed to estimate this parameter.
Also, keep in mind that, because of the sign in the normalizing
constant from the truncated distribution, positive values of
$\nu_{\gamma}$ are weighted more heavily than negative values, even
though all values of $\gamma_i$ in this model are negative.  This may
prove to be problematic as this may result in the chain failing to
converge on a final estimate, possibly leading to non-identifiability
in the $\gamma_i$ and $\delta_{\gamma}$ posteriors.\\

$\delta_{\gamma}$:\\
\begin{eqnarray*}
f(\delta_{\gamma}|\dots)&\propto&(\delta_{\gamma})^{-\frac{I}{2}}\exp\left(-\frac{\sum_i(\gamma_i-\nu_{\gamma})^2}{2\delta_{\gamma}}\right)\Phi\left(\frac{-\nu_{\gamma}}{\sqrt{\delta_{\gamma}}}\right)^{-I}\delta_{\gamma}^{-a_{\gamma}-1}\exp\left(-\frac{b_{\gamma}}{\delta_{\gamma}}\right)\\
&=&\delta_{\gamma}^{-\frac{I}{2}-a_{\gamma}-1}\exp\left(-\frac{1}{\delta_{\gamma}}\left[\frac{\sum_i(\gamma_i-\nu_{\gamma})^2}{2}+b_{\gamma}\right]\right)\Phi\left(\frac{-\nu_{\gamma}}{\sqrt{\delta_{\gamma}}}\right)^{-I}\\
\end{eqnarray*}
This distribution is not an Inverse-Gamma.  This, the Inverse-Gamma
distribution is not a conjugate prior for this parameter.  This
distribution is also not of a commonly known form, as no common
distribution has a density composed of a product of an Inverse-Gamma
density function scales by a Standard Normal CDF.  Consequently,
Metropolis-Hastings sampling is needed to estimate this parameter.

%Although the resulting posterior for $\gamma_i$ is also a
%Truncated-Normal and, thus, implies that the Truncated-Normal is a
%conjugate prior for $\gamma_i$, this does not hold for $\nu_{\gamma}$
%or for $\delta_{\gamma}$.  More importantly, the posteriors for
%$\nu_{\gamma}$ and $\delta_{\gamma}$ are not of a commonly-known form
%and, consequently, do not have well-known normalizing constants.  As
%such, Metropolis-Hastings steps will be needed to estimate these
%parameters for this model.\\


\subsubsection{Coactive Model}

Aside from the support set for the $\gamma_i$ parameters, this model
is identical to the parallel model.  As such, the derivations are, for
the most part, identical.  The only difference is in the normalizing
constants used in the truncated-normal distributions, as illustrated
here:
\begin{eqnarray*}
f(\gamma_i)&=&(2\pi\delta_{\gamma})^{-\frac{1}{2}}\exp\left(-\frac{(\gamma_i-\nu_{\gamma})^2}{2\delta_{\gamma}}\right)\mathcal{I}\left(\gamma_i>0\right)\left(1-\Phi\left(\frac{0-\nu_{\gamma}}{\sqrt{\delta_{\gamma}}}\right)\right)^{-1}\\
&=&(2\pi\delta_{\gamma})^{-\frac{1}{2}}\exp\left(-\frac{(\gamma_i-\nu_{\gamma})^2}{2\delta_{\gamma}}\right)\mathcal{I}\left(\gamma_i>0\right)\Phi\left(\frac{\nu_{\gamma}}{\sqrt{\delta_{\gamma}}}\right)^{-1}\\
\end{eqnarray*}
with the values of 1 and 0 returned by the indicator function
corresponding to values of gamma greater than and not greater than 0,
respectively, and with the sign of the term in the Standard-Normal CDF
term switched.\\

Because of this similarity, only the posterior distributions are
listed here.  Note that this means the posterior distributions for
$\nu_{\gamma}$ and $\delta_{\gamma}$ in this model are also neither
conjugate nor of known form, necessitating Metropolis-Hastings steps
for estimation and potentially encountering the same failures.\\

$\gamma_i$:\\
\begin{eqnarray*}
\gamma_i|\dots&\sim&\mbox{Truncated-Normal}\left(\nu_{\gamma}',\delta_{\gamma}',\left\{0,\cdots,\infty\right\}\right)
\end{eqnarray*}

$\nu_{\gamma}$:\\
\begin{eqnarray*}
f(\nu_{\gamma}|\dots)&\propto&\exp\left(-\frac{1}{2}\left[\nu_{\gamma}^2\left(\frac{I}{\delta_{\gamma}}+\frac{1}{\theta_{\gamma}}\right)-2\nu_{\gamma}\left(\frac{I\bar{\gamma}}{\delta_{\gamma}}+\frac{\phi_{\gamma}}{\theta_{\gamma}}\right)\right]\right)\Phi\left(\frac{\nu_{\gamma}}{\sqrt{\delta_{\gamma}}}\right)^{-I}\\
\end{eqnarray*}

$\delta_{\gamma}$:\\
\begin{eqnarray*}
f(\delta_{\gamma}|\dots)&\propto&\delta_{\gamma}^{-\frac{I}{2}-a_{\gamma}-1}\exp\left(-\frac{1}{\delta_{\gamma}}\left[\frac{\sum_i(\gamma_i-\nu_{\gamma})^2}{2}+b_{\gamma}\right]\right)\Phi\left(\frac{-\nu_{\gamma}}{\sqrt{\delta_{\gamma}}}\right)^{-I}\\
\end{eqnarray*}

%\end{document}



%%%%%%%%%%%%%%%%%%%%%%%%%%%%%%%%%%%%%%%%%%%%%%%%%


\subsection{Gamma Model}
%% Gamma Model Derivation from Dissertation
%\documentclass[12pt]{report}
%\begin{document}

For $i = 1,...I$, $j = 1,2$, $k= 1,2$, $l = 1,...,L_{ijk}$, $m = 1,...,4$,
the common assumptions are as follows:\\
\begin{eqnarray*}
y_{ijk\ell} &\sim& \mbox{Gamma}\left(\theta,\mu_{ijk}\right)\\
\mu_{ijk} &=& \eta_i + \alpha_i\left(-1\right)^j + \beta_i 
\left(-1\right)^k + \gamma_i \left(-1\right)^{j+k}\\
\theta &\sim& \mbox{Gamma}(\kappa_\theta,\tau_\theta)\\
\eta_i &\sim& \mbox{Gamma}(\kappa_\eta,\tau_\eta)\\
\alpha_i &\sim& \mbox{Normal}(\nu_\alpha,\epsilon_\alpha)\\
\beta_i &\sim& \mbox{Normal}(\nu_\beta,\epsilon_\beta)\\
%\gamma_i &\sim& \mbox{GM}(m,\nu_\gamma,\epsilon_\gamma)\\
\nu_\alpha &\sim& \mbox{Normal}(\zeta_\alpha,\lambda_\alpha)\\
\nu_\beta &\sim& \mbox{Normal}(\zeta_\beta,\lambda_\beta)\\
%\nu_\gamma &\sim& \mbox{Normal}(\zeta_\gamma,\lambda_\gamma)\\
\epsilon_\alpha &\sim& \mbox{Inverse-Gamma}(\kappa_\alpha,\tau_\alpha)\\
\epsilon_\beta &\sim& \mbox{Inverse-Gamma}(\kappa_\beta,\tau_\beta)\\
%\epsilon_\gamma &\sim& \mbox{Inverse-Gamma}(\kappa_\gamma,\tau_\gamma)\\
\end{eqnarray*}
%where $\mbox{GM}(m,\nu_\gamma,\epsilon_\gamma)$ is defined as follows:\\
%\begin{eqnarray*}
%\mbox{GM}\left(m,\nu_\gamma,\epsilon_\gamma\right) &=&\left\{
%\begin{array}{ccl}
%\mbox{Degenerate}(0)&\mbox{for}&m=1\\
%\mbox{Normal}\left(\nu_\gamma,\epsilon_\gamma\right)&\mbox{for}&m=2\\
%\mbox{Truncated-Normal}\left(\nu_\gamma,\epsilon_\gamma,(-\infty,0
%]\right)&\mbox{for}&m=3\\
%\mbox{Truncated-Normal}\left(\nu_\gamma,\epsilon_\gamma,[0,\infty
%)\right)&\mbox{for}&m=4\\
%\end{array}\right.
%\end{eqnarray*}
In this framework, $\eta_i$ follows a Gamma distribtion instead of a
Normal distribution as an attempt to keep the (typically largest
portion of the) scale parameter positive.  However, aside from making
a slight change to how $\eta_i$ is sampled, this has no substantial
effect on the other parameters.
%\newpage

As in the Normal framework, the specific model assumptions are as follows:

\subsubsection{Serial Model}
Because the MIC predicted by serial processing is zero, this model
assumes that all interaction parameters, $\gamma_i$ are exactly zero.
Because these parameters are all constant, there is no need for any
further hierarchical specification in this model.

\subsubsection{General Model}
The general model places minimal constraint on the possible range of
the interaction parameters.  For the purposes of this current
research, we will assume that the interaction parameters each follow a
normal distribution with a hierarchical structure described as follows:
\begin{eqnarray*}
\gamma_i &\sim& \mbox{Normal}\left(\nu_\gamma,\epsilon_\gamma\right)\\
\nu_\gamma &\sim& \mbox{Normal}\left(\zeta_\gamma,\lambda_\gamma\right)\\
\epsilon_\gamma &\sim& \mbox{Inverse-Gamma}\left(\kappa_\gamma,\tau_\gamma\right)\\
\end{eqnarray*}
where $\zeta_\gamma$, $\lambda_\gamma$, $\kappa_\gamma$, and
$\tau_\gamma$ are hyperparameters that are specified before analysis.
It should be noted that the assumptions in this model are very similar
to those in the next two models.  The only major change is in the
distribution of the interaction parameter itself, with the hyperpriors
on $\nu_\gamma$ and $\epsilon_\gamma$ remaining the same.  As such,
only the change to the distribution to $\gamma_i$ will be noted.

\subsubsection{Parallel Model}
Because parallel processing predicts negative MICs, this model assumes
that all interaction parameters must be negative.  This is
accomplished by the following restriction:
\begin{eqnarray*}
\gamma_i &\sim& \mbox{Truncated-Normal}\left(\nu_\gamma,\epsilon_\gamma,(-\infty,0]\right)
\end{eqnarray*}
with $(-\infty,0]$ being the support set for these truncated
distributions.  This limitation to the support set constitutes the
sign constraint needed for this model and only allows interaction
parameters to equal zero in order to allow for Savage-Dickey
estimation.

\subsubsection{Coactive Model}
Because coactive processing predicts positive MICs, this model assumes
that all interaction parameters must be positive.  This is
accomplished by the following restriction:
\begin{eqnarray*}
\gamma_i &\sim& \mbox{Truncated-Normal}\left(\nu_\gamma,\epsilon_\gamma,[0,\infty)\right)
\end{eqnarray*}
with $[0,\infty)$ being the support set for these truncated
distributions.  This limitation to the support set constitutes the
sign constraint needed for this model and only allows interaction
parameters to equal zero in order to allow for Savage-Dickey
estimation.


\subsubsection{Joint Distribution and Posterior Derivation of Common Parameters}
Because the common parameters are independent of the specific
parameters and only depend on them when they are assumed to be known,
the priors for the model-specific parameters are effectively
normalizing constants that can be dropped during derivation up to
proportionality.  As such, those distributions are of no interest to
the current joint analysis and will be omitted.  Consequently, the
joint distribution of all common parameters is as follows:\\
%\begin{center}
\begin{eqnarray*}
&\displaystyle{f\left(Y,\theta,\vec{\eta},\vec{\alpha},\vec{\beta},\nu_\alpha,\nu_\beta,\epsilon_\alpha,\epsilon_\beta\right)}&\\
&\displaystyle{=\left(\prod_i\prod_j\prod_k\prod_l(\mu_{ijk})^{-\theta}\Gamma(\theta)^{-1}(y_{ijk\ell})^{\theta-1}\exp\left(-\frac{y_{ijk\ell}}{\mu_{ijk}}\right)\right)\times}&\\
&\displaystyle{(\tau_\theta)^{-\kappa_\theta}\Gamma(\kappa_\theta)^{-1}(\theta)^{\kappa_\theta-1}\exp\left(-\frac{\theta}{\tau_\theta}\right)\times\left(\prod_i(\tau_\eta)^{-\kappa_\eta}\Gamma(\kappa_\eta)^{-1}(\eta_i)^{\kappa_\eta-1}\exp\left(-\frac{\eta_i}{\tau_\eta}\right)\right)\times}&\\
&\displaystyle{\left(\prod_i(2\pi\epsilon_\alpha)^{-\frac{1}{2}}\exp\left(-\frac{(\alpha_i-\nu_\alpha)^2}{2\epsilon_\alpha}\right)\right)\times\left(\prod_i(2\pi\epsilon_\beta)^{-\frac{1}{2}}\exp\left(-\frac{(\beta_i-\nu_\beta)^2}{2\epsilon_\beta}\right)\right)\times}&\\
&\displaystyle{(2\pi\lambda_\alpha)^{-\frac{1}{2}}\exp\left(-\frac{(\nu_\alpha-\zeta_\alpha)^2}{2\lambda_\alpha}\right)\times(2\pi\lambda_\beta)^{-\frac{1}{2}}\exp\left(-\frac{(\nu_\beta-\zeta_\beta)^2}{2\lambda_\beta}\right)\times}&\\
&\displaystyle{(\tau_\alpha)^{\kappa_\alpha}\Gamma(\kappa_\alpha)^{-1}(\epsilon_\alpha)^{-\kappa_\alpha-1}\exp\left(-\frac{\tau_\alpha}{\epsilon_\alpha}\right)\times(\tau_\beta)^{\kappa_\beta}\Gamma(\kappa_\beta)^{-1}(\epsilon_\beta)^{-\kappa_\beta-1}\exp\left(-\frac{\tau_\beta}{\epsilon_\beta}\right)\times}&\\
\end{eqnarray*}
%\end{center}

Unfortunately, none of the prior distributions are conjugate priors,
nor are most of the hyperprior distributions.  More importantly, most
of the conditional posterior distributions do not have a
frequently-used functional form, as can be seen from the example
below.  Consequently, most of the chains for this framework will rely
largely on Metropolis steps to generate adequate samples.

An example of the general lack of a clean functional form can be seen
in the posterior distribution of $\alpha_i$:
\begin{eqnarray*}
\displaystyle{f(\alpha_i|\cdot)}&\propto&\displaystyle{\left(\prod_j\prod_k\prod_l(\mu_{ijk})^{-\theta}\exp\left(-\frac{y_{ijk\ell}}{\mu_{ijk}}\right)\right)\times\exp\left(-\frac{(\alpha_i-\nu_\alpha)^2}{2\epsilon_\alpha}\right)}\\
&=&\displaystyle{\left(\prod_j\prod_k\prod_l(\eta_i + \alpha_i\left(-1\right)^j + \beta_i \left(-1\right)^k + \gamma_i \left(-1\right)^{j+k})^{-\theta}\right.}\\
&&\displaystyle{\left.\exp\left(-\frac{y_{ijk\ell}}{\eta_i + \alpha_i\left(-1\right)^j + \beta_i \left(-1\right)^k + \gamma_i \left(-1\right)^{j+k}}\right)\right)\times}\\
&&\displaystyle{\exp\left(-\frac{(\alpha_i-\nu_\alpha)^2}{2\epsilon_\alpha}\right)}\\
\end{eqnarray*}
The main problem is the $(\mu_{ijk})^{-\theta}$ term, which becomes
more complex as $\theta$ increases and, for $\theta \neq 1$, prevents
separation of any component from the others.  This might not be as
much of an issue if $\theta$ is held constant, which will be done in
current work as earlier Gamma framework development has found that
chain estimates of the $\theta$ parameter seem to climb until the
sample space for the parameters of interest becomes too small to
analyze, but derivation of the posterior distribution was still done
in the interest of future refinement.  Because this is a common issue
for all parameters (and because the other parameters have similarly
non-standard posterior distributions), the posterior distributions for
most parameters are omitted here for brevity and, along with the
analysis code, will be included in supplementary materials for any
interested parties.  However, the posterior distributions for the
model-specific interaction terms will be included subsequently for
reference.

\subsubsection{Specific Posterior Derivations}
The posterior distributions for the hierarchical paramters,
$\nu_\gamma$ and $\epsilon_\gamma$ in the Gamma framework are
effectively identical to their Normal framework counterparts,
$\nu_\gamma$ and $\delta_\gamma$, respectively, as the specific
distributions involved are the same.  Consequently, the reader is
referred to their derivation in the previous chapter, and only the
posterior distribution for the interaction parameters themselves will
be included here.\\

\subsubsection{Serial Model}
Just like in the Normal framework, the Serial model assumes that all
interaction parameters, $\gamma_i$, are equal to 0.  As such, there is
no need to estimate either $\nu_\gamma$ or $\epsilon_\gamma$, as they
effectively do not exist for this model.\\

\subsubsection{General Model}
In this model, the prior distribution over each $\gamma_i$ parameter
is similar to the priors for the $\alpha_i$ (described above) and
$\beta_i$ parameters.  As such, the posterior derivations for
$\gamma_i$ are also similar to the posteriors for $\alpha_i$ and
$\beta_i$:

%JON - DERIVATION
\begin{eqnarray*}
\displaystyle{f(\gamma_i|\cdot)}&\propto&\displaystyle{\left(\prod_j\prod_k\prod_l(\mu_{ijk})^{-\theta}\exp\left(-\frac{y_{ijk\ell}}{\mu_{ijk}}\right)\right)\times\exp\left(-\frac{(\gamma_i-\nu_\gamma)^2}{2\epsilon_\gamma}\right)}\\
&=&\displaystyle{\left(\prod_j\prod_k\prod_l(\eta_i + \alpha_i\left(-1\right)^j + \beta_i \left(-1\right)^k + \gamma_i \left(-1\right)^{j+k})^{-\theta}\right.}\\
&&\displaystyle{\left.\exp\left(-\frac{y_{ijk\ell}}{\eta_i + \alpha_i\left(-1\right)^j + \beta_i \left(-1\right)^k + \gamma_i \left(-1\right)^{j+k}}\right)\right)\times}\\
&&\displaystyle{\exp\left(-\frac{(\gamma_i-\nu_\gamma)^2}{2\epsilon_\gamma}\right)}\\
\end{eqnarray*}
This derivation is not extended much further because, as described
earlier, there is no simple expansion of the $(\mu_{ijk})^{-\theta}$
term when $\theta$ is not constant.  Additionally, because this term
is multiplied across iterations within conditions, the actual scale of
the exponent is substantially larger than $\theta$, further
complicating any attempt at expansion.  However, it must be noted that
the un-normalized posterior density is still easy to calculate as it
is effectively proportional to the product of a normal density and a
series of gamma densities, with all parameters known.


\subsubsection{Parallel Model}
In this model, all $\gamma_i$ parameters follow a Truncated-Normal
distribution that is limited between $-\infty$ and 0.  The density for
$\gamma_i$ in this case is:

\begin{eqnarray*}
\displaystyle{f(\gamma_i|\cdot)}&\propto&\displaystyle{\left(\prod_j\prod_k\prod_l(\mu_{ijk})^{-\theta}\exp\left(-\frac{y_{ijk\ell}}{\mu_{ijk}}\right)\right)\times\exp\left(-\frac{(\gamma_i-\nu_\gamma)^2}{2\epsilon_\gamma}\right)\times\mathcal{I}\left(\gamma_i<0\right)}\\
&=&\displaystyle{\left(\prod_j\prod_k\prod_l(\eta_i + \alpha_i\left(-1\right)^j + \beta_i \left(-1\right)^k + \gamma_i \left(-1\right)^{j+k})^{-\theta}\right.}\\
&&\displaystyle{\left.\exp\left(-\frac{y_{ijk\ell}}{\eta_i + \alpha_i\left(-1\right)^j + \beta_i \left(-1\right)^k + \gamma_i \left(-1\right)^{j+k}}\right)\right)\times}\\
&&\displaystyle{\exp\left(-\frac{(\gamma_i-\nu_\gamma)^2}{2\epsilon_\gamma}\right)\times\mathcal{I}\left(\gamma_i<0\right)}\\
\end{eqnarray*}

Aside from the indicator function used to enforce the restriction, the
posterior distribution is effectively identical to that of the General
Model.  Because of this, this density can be calculated in roughly the
same manner.


\subsubsection{Coactive Model}
Aside from the support set for the $\gamma_i$ parameters, this model
is identical to the parallel model.  As such, the derivations are, for
the most part, identical.  The only differences are in the indicator
function and Normal-distribution CDF used in the truncated-normal
distributions, with only the indicator being included in the posterior
distribution as the CDF correction does not explicitly depend on
$\gamma_i$ and, consequently, is constant with respect to it.  Because
of this similarity, only the posterior distribution is listed here.

\begin{eqnarray*}
\displaystyle{f(\gamma_i|\cdot)}&\propto&\displaystyle{\left(\prod_j\prod_k\prod_l(\eta_i + \alpha_i\left(-1\right)^j + \beta_i \left(-1\right)^k + \gamma_i \left(-1\right)^{j+k})^{-\theta}\right.}\\
&&\displaystyle{\left.\exp\left(-\frac{y_{ijk\ell}}{\eta_i + \alpha_i\left(-1\right)^j + \beta_i \left(-1\right)^k + \gamma_i \left(-1\right)^{j+k}}\right)\right)\times}\\
&&\displaystyle{\exp\left(-\frac{(\gamma_i-\nu_\gamma)^2}{2\epsilon_\gamma}\right)\times\mathcal{I}\left(\gamma_i>0\right)}\\
\end{eqnarray*}

Again, because this is proportional to a product of a few well-known
distributions, this functional form is relatively easy to calculate,
even though the normalizing constant is not known.

%\end{document}

